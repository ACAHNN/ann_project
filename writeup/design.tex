The artificial neural network was implemented in Python.
It consists of a single class which takes as arguments (i) the number of input units (ii) a list containing the number hidden units to create at each hidden layer (where the length of the list defines the number of hidden layers to create (iii) the number of output units (iv) the learn rate and (v) the number of epochs to train the neural network for.
The neural network unit activations, errors, and weights are represented simply as matrices.
This allows all computations to be down in a fast and efficient manner and makes the neural network implementation very comphrensible.
The neural network class exports functions to (i) feed a single instance through the network (ii) run backpropagation (iii) train on a set of data (iv) test a set of data and (v) a set of functions to print weights, errors, and activations for debugging from the command line.
 

